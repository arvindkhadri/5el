\documentclass[''urlcolor=red'']{beamer}
\usepackage{hyperref}
\usetheme{Warsaw}
\title{Towards Social Semantic tools}
\author{Arvind}
\date{July 28, 2012}
\begin{document}

\begin{frame}
  \titlepage
\end{frame}

\begin{frame}{Towards Social Semantic tools}
  \begin{block}{Big data}
    \begin{itemize}
    \item Every single blog-post/tweet/status update that we make adds up.  The \textbf{big} conversations \textbf{data}.
    \item These conversations on the Web are hard to inter-connect.  For one thing they are not happening in markup language.  Say for example you tweet a link to a friend saying that the phone number in a page is wrong and give the correct number.
    \item The experience of Web will be enriched if when a person, maybe a friend of yours, at the page can see the correct number.
    \end{itemize}
  \end{block}
\end{frame}

\begin{frame}{Towards Social Semantic tools}
  \begin{block}{Implementation}
    \begin{itemize}
    \item A browser (extension) can help users in marking and in consuming annotations for a page.
    \item User A marks an element Bar on a page Foo and emits (``tweets'') an annotation.
    \item User B can experience the effect of the Foo\#Bar ``tweet'', say when on page Foo.
    \item Consider \url{http://jace.zaiki.in/2012/06/26/technology-outsource-vs-open-source}, where the author says for discussion go to a different link.
    \item A user would be able to see the comments on that page itself, rather than having to go to another page, by turning on friends' annotations.
    \end{itemize}
  \end{block}
\end{frame}

\begin{frame}{Towards Social Semantic tools}
  \begin{block}{Demo}
    \begin{itemize}
    \item \url{http://jace.zaiki.in/2012/06/26/technology-outsource-vs-open-source}
    \item \url{http://dev.a11y.in/web/feeds}
    \item \url{http://dev.a11y.in/web/?foruri=http://jace.zaiki.in/2012/06/26/technology-outsource-vs-open-source&tags=gplv3,\%20freedom&interactive=0&type=5el}
    \end{itemize}
  \end{block}
\end{frame}


\begin{frame}{Towards Social Semantic tools}
  \begin{block}{Alipi a re-narration web}
    \begin{itemize}
    \item The idea has been implemented in \url{http://alipi.us}
    \item If one translates a para or re-narrates to audio and video some content on an English page to Hindi.
    \item A Hindi aware person can read/listen/see the re-narration when visiting that page.
    \item \url{http://alipi.us}
    \item \url{http://y.a11y.in/web/?foruri=http\%3A\%2F\%2Fa11y.in\%2Fa11y_dw\%2F&lang=Hindi&interactive=1}
    \item \url{http://folkhampi.openrun.net/muraldemo.html}
    \end{itemize}
 \end{block}
\end{frame}

\begin{frame}{Towards Social Semantic tools}
  \begin{block}{Extending the idea towards digital heritage}
    \begin{itemize}
    \item Look at the mural \url{http://folkhampi.openrun.net/muraldemo.html}.
    \item What if someone comments on twitter saying ``Hey look this mural is similar to the ones found in Egypt.''.
    \item During a virtual walk comments/annotations show up if present.
    \end{itemize}
  \end{block}
\end{frame}

\begin{frame}{Thank You}
  \center @arvindkhadri about:semanticweb\#tools type:help request:contribute
  \begin{itemize}
  \item \url{http://janastu.org}
  \item \url{https://github.com/arvindkhadri}
  \item arvind@servelots.com
  \end{itemize}
  
\end{frame}

\end{document}
